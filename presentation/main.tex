\documentclass[10pt,sans,usenames,dvipsnames,english,compress]{beamer}

\usepackage[french]{babel}
\usepackage[utf8]{inputenc}
\usepackage[T1]{fontenc}
\usepackage{lmodern}
\usepackage{graphicx}
\usepackage[absolute,overlay]{textpos}
\usepackage{setspace}
\usepackage{listings}
\usepackage{multicol}
\usepackage{etoolbox}

\mode<presentation>

\usetheme{Warsaw}
\useoutertheme[subsection=false]{miniframes}

% Fix the crop bullets in the headline
\makeatletter
\setbeamertemplate{headline}
{
  \vskip-0.8ex
  \begin{beamercolorbox}{section in head/foot}
  \vskip2pt\insertnavigation{\paperwidth}\vskip4pt
  \end{beamercolorbox}%
}
\makeatother

%infolines footer
\defbeamertemplate*{footline}{infolines theme}
{
	\leavevmode%
	\hbox{%
	\begin{beamercolorbox}[wd=.333333\paperwidth,ht=2.25ex,dp=1ex,center]{author in head/foot}%
	 \usebeamerfont{author in head/foot}\insertshortauthor%~~(\insertshortinstitute)
	\end{beamercolorbox}%
	\begin{beamercolorbox}[wd=.333333\paperwidth,ht=2.25ex,dp=1ex,center]{title in head/foot}%
	 \usebeamerfont{title in head/foot}\insertshorttitle
	\end{beamercolorbox}%
	\begin{beamercolorbox}[wd=.333333\paperwidth,ht=2.25ex,dp=1ex,right]{date in head/foot}%
	 \usebeamerfont{date in head/foot}\insertshortdate{}\hspace*{2em}
	 \insertframenumber{} / \inserttotalframenumber\hspace*{2ex}
	\end{beamercolorbox}}%
	\vskip0pt%
}

\graphicspath{{images/}}

\setbeamersize{text margin left=15pt,text margin right=15pt}

% Disable navigation icons
\beamertemplatenavigationsymbolsempty

% Titlepage
\title{Analyse des données textuelles d'articles de presse}
\subtitle{Présentation Stage Première Année}
\author[Adib HABBOU]{Adib HABBOU}
\date{17 octobre 2022}
\institute[ENSIIE]{Haut-Commissariat au Plan du Maroc}

% Line vertical spacing
\setstretch{1.2}

% See for colors names: https://en.wikibooks.org/wiki/LaTeX/Colors

\lstdefinelanguage{yaml}{
keywords={true,false,null,y,n},
	keywordstyle=\color{darkgray}\bfseries,
	basicstyle=\color{black}\bfseries\tiny,
	sensitive=false,
	numbers=left,
	numbersep=5pt,
	numberstyle=\tiny\color{gray},
	comment=[l]{\#},
	morecomment=[s]{/*}{*/},
	commentstyle=\color{purple}\ttfamily,
	stringstyle=\color{blue}\mdseries\ttfamily,
	moredelim=[l][\color{orange}]{\&},
	moredelim=[l][\color{magenta}]{*},
	moredelim=**[il][\color{red}\mdseries{:}\color{blue}\mdseries]{:},
	morestring=[b]',
	morestring=[b]",
	literate =
		{---}{{\ProcessThreeDashes}}3
		{>}{{\textcolor{red}\textgreater}}1     
		{|}{{\textcolor{red}\textbar}}1 
		{\ -\ }{{\mdseries\ -\ }}3,
}

\lstset{
	language=java,
	basicstyle=\normalsize\normalfont\ttfamily\tiny,
	upquote=true,
	aboveskip={1\baselineskip},
	columns=fullflexible,
	showstringspaces=false,
	extendedchars=true,
	breaklines=true,
	showtabs=false,
	showspaces=false,
	showstringspaces=false,
	identifierstyle=\ttfamily,
	keywordstyle=\color[rgb]{0,0,1},
	commentstyle=\color[rgb]{0.133,0.545,0.133},
	stringstyle=\color[rgb]{0.627,0.126,0.941},
	numbers=left,
	numbersep=5pt,
	numberstyle=\tiny\color{gray},
	rulecolor=\color{black},
}

\begin{document}

\begin{frame}[plain]
	\vspace{0.6cm}
	\begin{minipage}{\textwidth}
		\centering
		\raisebox{-0.5\height}{\includegraphics[width=0.25\textwidth]{logo hcp vertical.png}}
		\hspace*{2cm}
		\raisebox{-0.5\height}{\includegraphics[width=0.25\textwidth]{logo ensiie.png}}
	\end{minipage}
	\vspace{0.6cm}
	\titlepage
\end{frame}

\section{Haut-Commissariat au Plan}
\begin{frame}{Haut-Commissariat au Plan}
	\begin{textblock*}{3cm}(9cm,2cm) % {block width} (coords)
		\includegraphics[width=3.5cm]{logo hcp horizontal.png}
	\end{textblock*}
 
        \vspace{1cm}
        
	\begin{itemize}
		\item \textbf{Structure ministérielle} marocaine créée en septembre 2003
		\item Principal producteur de l’\textbf{information statistique} au Maroc
            \item Locaux dans les quartiers d'\emph{Agdal} et de \emph{Hay Riad} à Rabat
		\item Admet une \textbf{Direction des Systèmes d’Information Statistique}
	\end{itemize}
\end{frame}

\begin{frame}{Direction des Systèmes d’Information Statistique}
	\begin{textblock*}{3cm}(9cm,2cm) % {block width} (coords)
		\includegraphics[width=3.5cm]{logo hcp horizontal.png}
	\end{textblock*}

	\vspace{1cm}

	\begin{itemize}
		\item \textbf{Collecte des données} à travers différentes sources
            \item \textbf{Nettoyage}, \textbf{classification} et \textbf{tri} des données
            \item \textbf{Stockage} des données dans les \textbf{Bases de Données Statistiques}
            \item \textbf{Analyse des données} pour extraire des \textbf{indicateurs}
            \item Publication de \textbf{rapports} \emph{(trimestriel, semestriel et annuel)}
	\end{itemize}
\end{frame}

\section{Contexte}
\begin{frame}{Objectifs}
        \begin{textblock*}{3cm}(1cm,2cm)
		\includegraphics[width=2.2cm]{mwn.png}
	\end{textblock*}

	\begin{textblock*}{3cm}(4.6cm,1.8cm)
		\includegraphics[width=3.5cm]{matin.png}
	\end{textblock*}

        \begin{textblock*}{3cm}(9.3cm,2.1cm)
		\includegraphics[width=2.8cm]{hespress.jpg}
	\end{textblock*}

        \vspace{1.5cm}

	\begin{itemize}
		\item Récupération et analyse d'articles de \textbf{presse en ligne marocaine}
            \vspace{0.3cm}
            \begin{enumerate}
                \item \normalsize{\emph{Morocco World News} $\rightarrow$ Anglais}
                \item \normalsize{\emph{Le Matin} $\rightarrow$ Français}
                \item \normalsize{\emph{Hespress} $\rightarrow$ Arabe}
            \end{enumerate}
        \end{itemize}
\end{frame}

\begin{frame}{Problématique}
	\textbf{Besoin de données} sur la réalité économique, sociale et culturelle du pays
	\begin{itemize}
		\visible<1->{\item[$\rightarrow$] {Études \textbf{ministérielles} et \textbf{gouvernementales}}}
		\visible<1->{\item[$\rightarrow$] {Études de marché pour le \textbf{secteur privé}}}
            \visible<1->{\item[$\rightarrow$] {Études \textbf{académiques} et \textbf{universitaires}}}
	\end{itemize}

	\visible<2->{
		\begin{block}{Existant}
			\begin{itemize}
				\item Observatoire des conditions de vie des ménages
				\item Centre d’études et de recherches démographiques
                \item Recensement général de la population et de l'habitat
			\end{itemize}
		\end{block}
	}

	\visible<3->{
		\begin{exampleblock}{Solutions}
			\begin{itemize}
				\item Web Scraping
				\item Machine Learning
			\end{itemize}
		\end{exampleblock}
	}
\end{frame}

\section{Web Scraping}
\begin{frame}{Web Scraping}
	\begin{textblock*}{3cm}(9.5cm,1.5cm) % {block width} (coords)
		\includegraphics[width=2.5cm]{logo selenium.png}
	\end{textblock*}

	\vspace{1cm}

        \begin{block}{}
		Technique de récupération de données à partir d'un site Web dynamique
	\end{block}

        \vspace{0.5cm}

	\visible<2->{\begin{itemize}
		      \item \textbf{Interaction} avec les navigateurs via un driver
            \item \textbf{Navigation} sur une page web dynamique (formulaires, boutons...) 
            \item \textbf{Localisation} des éléments (id, name, xpath, class name...)
	\end{itemize}}
\end{frame}

\begin{frame}{Web Scraping}
	\begin{exampleblock}{Méthodologie adoptée}
		\begin{enumerate}
             \item Parcours des \textbf{catégories} : https://www.site.com/category
            \item Parcours des \textbf{pages} : https://www.site.com/category/page\_number
            \item Extraction des \textbf{liens} de chaque article
            \item Parcours des liens d'articles et \textbf{récupération des données}
            \end{enumerate}
	\end{exampleblock}
\end{frame}

\begin{frame}{Web Scraping}
\begin{figure}[!h]
    \centering
    \includegraphics[scale=0.3]{inspect(1).png}
    \caption{Capture d'écran du site Morocco World News}
\end{figure}
\end{frame}

\begin{frame}{Web Scraping - Résultat}
    \begin{figure}[!h]
    \centering
    \includegraphics[scale=0.4]{df_scrap.png}
    \caption{Data Frame de Morocco World News}
    \end{figure}
\end{frame}

\begin{frame}{Web Scraping - Résultat}
    \begin{figure}[!h]
    \centering
    \includegraphics[scale=0.6]{df_scrap_cat.png}
    \caption{Data Frame de Morocco World News}
    \end{figure}
\end{frame}

\section{Topic Modeling}
\begin{frame}{Topic Modeling}
	\begin{textblock*}{3cm}(1cm,2.2cm)
		\includegraphics[width=3.5cm]{logo gensim.png}
	\end{textblock*}

	\begin{textblock*}{3cm}(9.5cm,1.5cm)
		\includegraphics[width=2cm]{logo nltk.png}
	\end{textblock*}

	\vspace{1cm}

	\begin{block}{}
		Modèle non supervisé permettant d'extraire les sujets de documents
	\end{block}

        \vspace{0.5cm}

	\visible<2->{\begin{itemize}
		\item Preprocessing des données avec \textbf{NLTK}
            \item Application du modèle avec \textbf{Gensim}
	\end{itemize}}
\end{frame}

\begin{frame}{Topic Modeling}
\begin{figure}[!h]
    \centering
    \includegraphics[scale=0.45]{tm(1).png}
    \caption{Schéma Topic Model}
\end{figure}
\end{frame}

\begin{frame}{Topic Modeling - LDA}
	\begin{exampleblock}{Latent Dirichlet Allocation}
		\begin{itemize}
            \item Chaque document est un \textbf{mélange} $\theta$ d’un petit nombre de sujets
            \item Attribue un \textbf{topic} à chaque \textbf{document} selon une distribution de Dirichlet : $$\theta_{i} \sim Dir(\alpha) \: \: pour \: \: 1 \leq i \leq M \: \: avec \: \: \alpha < 1$$
            \item \textbf{Mise à jour} du topic lié à chaque document en fonction de la probabilité : $$\mathcal{P}(t | d) \: \: x \: \: \mathcal{P}(w | t)$$
          \end{itemize}
	\end{exampleblock}

        \visible<2->{\begin{block}{Calcul de probabilité}
		\begin{itemize}
            \item $\mathcal{P}(t | d)$ : la probabilité que le \textbf{document} $d$ soit assigné au \textbf{topic} $t$
            \item $\mathcal{P}(w | t)$ : la probabilité que le \textbf{topic} $t$ soit assigné au \textbf{mot} $w$
            \end{itemize}
	\end{block}}
\end{frame}

\begin{frame}{Topic Modeling - LDA}
	\begin{exampleblock}{Preprocessing}
            \begin{itemize}
            \item \textbf{Tokenisation} : tout en minuscule, suppression de la ponctuation
            \item \textbf{Lemmatisation} : tout au présent et 1ère personne
            \item \textbf{Racinisation} : réduction à la forme radicale
            \item \textbf{Bag of Words} : dictionnaires \{mots : nombre d’occurrences\}
            \end{itemize}
	\end{exampleblock}

        \visible<2->{\begin{block}{Visualisation \textbf{pyLDAvis}}
		\begin{itemize}
            \item Extraction des informations du \textbf{topic model}
            \item Réalisation d'une visualisation \textbf{Web interactive}
            \end{itemize}
	\end{block}}
\end{frame}

\begin{frame}{Topic Modeling - Résultat}
    \begin{figure}[!h]
    \centering
    \includegraphics[scale=0.35]{pyLDAvis.png}
    \caption{Capture d'écran résultat pyLDAvis}
    \end{figure}
\end{frame}

\section{Text Classification}
\begin{frame}{Text Classification}
	\begin{textblock*}{3cm}(9cm,2cm)
		\includegraphics[width=3cm]{logo sklearn.png}
	\end{textblock*}

	\vspace{1cm}

	\begin{block}{}
		Modèle supervisé nécessitant un entraînement sur des données labélisées
	\end{block}

        \vspace{0.5cm}

	\visible<2->{\begin{itemize}
		      \item \textbf{Preprocessing} des données avec \textbf{NLTK}
            \item \textbf{Application} de différents modèles avec \textbf{Scikit-Learn}
            \item \textbf{Évaluation} des modèles de Machine Learning avec \textbf{Scikit-Learn}
	\end{itemize}}
\end{frame}

\begin{frame}{Text Classification}
	\begin{exampleblock}{Modèles de Machine Learning}
            \begin{itemize}
            \item K Plus Proche Voisins
            \item Régression Logistique
            \item Forêt Aléatoire
            \item Machine à Vecteurs de Support
            \item Perceptron Multicouche
            \end{itemize}
	\end{exampleblock}

        \visible<2->{\begin{block}{Évaluation des modèles}
            \begin{itemize}
            \item \textbf{Précision} : taux de prédictions positives correctes
            \item \textbf{Recall} : taux de positifs correctement prédits
            \item \textbf{F1-score} : capacité d’un modèle à bien prédire les individus positifs
            \item \textbf{Matrice Confusion} : ligne catégorie réelle, colonne catégorie estimée
            \end{itemize}
	\end{block}}
\end{frame}

\begin{frame}{K Plus Proche Voisins}
\begin{figure}[!h]
    \centering
    \includegraphics[scale=0.42]{KNN(1).png}
    \caption{Une itération de l'algorithme KNN}
\end{figure}
\end{frame}

\begin{frame}{Régression Logistique}
\begin{figure}[!h]
    \centering
    \includegraphics[scale=0.4]{RGL(1).png}
    \caption{Régression Linéaire VS Régression Logistique}
\end{figure}
\end{frame}

\begin{frame}{Forêt Aléatoire}
\begin{figure}[!h]
    \centering
    \includegraphics[scale=0.32]{RF(1).png}
    \caption{Forêt Aléatoire de N Arbres de Décision}
\end{figure}    
\end{frame}

\begin{frame}{Machine à Vecteurs de Support}
\begin{figure}[!h]
    \centering
    \includegraphics[scale=0.43]{SVM(1).png}
    \caption{Représentation du SVM}
\end{figure}
\end{frame}

\begin{frame}{Perceptron Multicouche}
\begin{figure}[!h]
    \centering
    \includegraphics[scale=0.27]{MLP(1).png}
    \caption{Perceptron Multicouche}
\end{figure}
\end{frame}

\begin{frame}{Rapport de classification}
    \begin{figure}[!h]
    \centering
    \includegraphics[scale=0.7]{report.png}
    \caption{Rapport de classification pour le SVM}
    \end{figure}
\end{frame}

\begin{frame}{Matrice de confusion}
    \begin{figure}[!h]
    \centering
    \includegraphics[scale=0.35]{conf.png}
    \caption{Matrice de confusion pour le SVM}
    \end{figure}
\end{frame}

\section{Dashboard}
\begin{frame}{Dashboard}
	\begin{textblock*}{3cm}(9cm,2cm)
		\includegraphics[width=3cm]{logo streamlit.png}
	\end{textblock*}

	\vspace{1cm}

	\begin{block}{}
		Application Web pour présenter visuellement les résultats obtenus
	\end{block}

        \vspace{0.5cm}

	\visible<2->{\begin{itemize}
		\item \textbf{Transformation} de script Python en application Web avec \textbf{Streamlit}
            \item \textbf{Intégration} de code HTML généré par \textbf{pyLDAvis}
            \item \textbf{Intégration} de graphiques créés avec \textbf{Plotly}
	\end{itemize}}
\end{frame}

\begin{frame}{Dashboard - Résultat}
    \begin{figure}[!h]
    \centering
    \includegraphics[scale=0.25]{Dash 1.png}
    \caption{Capture d'écran du Dashboard}
    \end{figure}
\end{frame}

\begin{frame}{Dashboard - Résultat}
    \begin{figure}[!h]
    \centering
    \includegraphics[scale=0.2]{Dash 3.png}
    \caption{Capture d'écran du Dashboard}
    \end{figure}
\end{frame}

\begin{frame}{Dashboard - Résultat}
    \begin{figure}[!h]
    \centering
    \includegraphics[scale=0.2]{Dash 4.png}
    \caption{Capture d'écran du Dashboard}
    \end{figure}
\end{frame}

\section{Conclusion}
\begin{frame}{Conclusion}
	\begin{exampleblock}{Apport}
		\begin{itemize}
		\item Étudier la \textbf{faisabilité} de l’ensemble du processus \newline
  $\implies$ Web Scraping, Topic Modeling et Text Classification
		\item Identifier l’ensemble des \textbf{technologies} nécessaires \newline
  $\implies$ Selenium, NLTK, Gensim, Scikit-Learn, Plotly, WordCloud, Streamlit
		\item Sélectionner les \textbf{modèles} de Machine Learning les plus \textbf{performants} \newline
  $\implies$ Machine à Support de Vecteur et Régression Logistique
		\end{itemize}
	\end{exampleblock}
\end{frame}

\begin{frame}{Conclusion}
\begin{alertblock}{Perspective}
	\begin{itemize}
        \item Construction d'une \textbf{infrastructure d’extraction et de nettoyage} \newline
  $\implies$ Web Scraping \& Preprocessing
        \item Mise en place d'une \textbf{Data Analytics Pipeline} \newline
  $\implies$ Topic Modeling \& Text Classification
        \item Développement d'une \textbf{application Web de monitoring} des résultats \newline
  $\implies$ Afin de surveiller le modèle et pouvoir l'améliorer
	\end{itemize}
\end{alertblock}    
\end{frame}

\begin{frame}
    \centering
    \LARGE{Merci pour votre attention !}
\end{frame}


\end{document}

